

\documentclass[conference]{IEEEtran}

\usepackage[utf8]{inputenc}
\usepackage[pdftex]{graphicx}
%\usepackage{unicode-math}
%\usepackage{mathtools}
\usepackage{amsmath}


\hyphenation{op-tical net-works semi-conduc-tor}


\begin{document}

\renewcommand{\abstractname}{Resumo}
\renewcommand{\refname}{REFERÊNCIAS}
\renewcommand{\tablename}{TABELA}


\title{Análise de Complexidade e Desempenho de Algoritmos de Ordenação}


% author names and affiliations
% use a multiple column layout for up to three different
% affiliations
%\author{\IEEEauthorblockN{Bruno Santos de Lima\\Leandro Ungari Cayres}
%\IEEEauthorblockA{Faculdade de Ciências e Tecnologia\\
%Universidade Estadual Paulista\\
%Presidente Prudente, Brasil\\
%\textit{leandroungari@gmail.com}}
%}

\author{ 
 Bruno Santos de Lima\\ 
 \IEEEauthorblockA{Faculdade de Ciências e Tecnologia\\
Universidade Estadual Paulista\\
Presidente Prudente, Brasil \\
 \textit{brunoslima@gmail.com}
} 
 \and 
 Leandro Ungari Cayres\\ 
 \IEEEauthorblockA{Faculdade de Ciências e Tecnologia\\
Universidade Estadual Paulista\\
Presidente Prudente, Brasil \\
 \textit{leandroungari@gmail.com} 
}
} 



% make the title area
\maketitle

% As a general rule, do not put math, special symbols or citations
% in the abstract
\begin{abstract}

O Índice de Desenvolvimento Humano busca mensurar o avanço na qualidade de vida de uma população, através de um cálculo matemático que envolve atributos relacionados a educação, saúde e renda. Esse índice tem sido utilizado nas diferentes esferas políticas de governo (municipal, estadual e federal) como apoio a adoção de políticas públicas e investimentos econômicos. A utilização de técnicas de visualização para a interpretação de dados desse contexto, consiste em uma importante ferramenta, pois permite uma apresentação sumarizada dos dados e viabiliza a identificação de padrões em determinadas regiões e a evolução destes conforme uma dada função. Neste trabalho, o processo exploratório foi realizado utilizando a técnica de visualização \textit{Choropleth Map}.

~\\Palavras-chave: Visualização Computacional, Índice de Desenvolvimento Humano e \textit{Choropleth Map}.

\end{abstract}


\IEEEpeerreviewmaketitle


\section{Introdução}

dasdlasjfsdfjskjfsfsj

\begin{thebibliography}{1}


\bibitem{atlas}
Programa das Nações Unidas para o Desenvolvimento (PNUD). \emph{Atlas do desenvolvimento humano do Brasil}. PNUD; 2003. Disponível em: http://www.pnud.org.br/atlas/

\bibitem{senak}
Sen AK. \emph{Desenvolvimento como liberdade}. São Paulo: Companhia das Letras; 2000.


\end{thebibliography}




% that's all folks
\end{document}


